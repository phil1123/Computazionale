\subsection*{Introduzione teorica}
Si vuole studiare il problema agli autovalori:
$$ -\nabla^2\psi + V(x)\psi = E \psi $$
dove si è posto $\hbar = 1$ e $m = 1$. Il potenziale è assegnato da:
$$ V(x)  = \begin{cases}
        0       & \mbox{se } x<-b \\
        4/b^2   & \mbox{se } -b \leq x \leq b \\
        0       & \mbox{se } x>b \\
         \end{cases}
$$
Siano:
$$ \begin{cases}
    \mbox{ZonaI} & = \{x<-b \} \\
    \mbox{ZonaII} & = \{-b \leq x \leq b\} \\
    \mbox{ZonaIII} & = \{x>b\}
\end{cases} $$
\\
Sia $k^2 = 2E$ e sia $q^2 = 2(E-V0)$. La soluzione analitica più generale è data da:
$$\psi_k(x) =
    \begin{cases}
        Ae^{ikx}+Be^{-ikx} & \mbox{in ZonaI} \\
        f(x) & \mbox{in ZonaII} \\
        Ce^{ikx}+De^{-ikx} & \mbox{in ZonaIII} \\
    \end{cases}
$$
Non siamo per il momento interessati alla ZonaII, quindi indichiamo con
$f(x)$ la $\psi_k(x)$ in tale zona, che a rigore sarebbe
    $$f(x)= Ee^{iqx}+Fe^{-iqx}$$
Le costanti $A,B,C,D$ sono determinate dalle condizioni di raccordo di continuità della
funzione d'onda e della sua derivata nei punti $x=\pm b$.\\
Si osserverà che dalla diagonalizzazione della versione discreta della Hamiltoniana
(vedi sezione apposita) risultano autofunzioni a parità definita (pari o dispari).
Allora, senza perdita di generalità, si pone la ulteriore condizione di simmetria alle $\psi_k$, che porta:
    $$ A = D \mbox{ se funzioni, e } A = -D \mbox{ se funzioni dispari}$$
Per la conservazione del flusso di probabilità, si possono riscrivere nella seguente forma:
$$
    \begin{pmatrix} C \\ B \end{pmatrix} =
    \begin{pmatrix} \tau & \rho \\ \rho & \tau \end{pmatrix} \cdot
    \begin{pmatrix} A \\ D \end{pmatrix} = \quad (S)\cdot \begin{pmatrix} A \\ D \end{pmatrix}
$$
Ove la matrice S è una matrice unitaria, ossia che verifica le condizioni:
$$ \tau\rho^* + \tau^*\rho = 0 \quad,\quad |\tau|^2 + |\rho|^2 = 1$$
(Si è indicato con $z^*$ il numero complesso coniugato di $z$). Segue immediatamente che:
$$ |\tau \pm \rho|^2 = |\tau|^2 + |\rho|^2 + \tau\rho^* + \tau^*\rho = |\tau|^2 + |\rho|^2 + 0 = 1$$
Cioè $\tau \pm \rho$ differiscono per una fase:
$$ |\tau \pm \rho|^2 = 1 \Rightarrow (\tau \pm \rho) = e^{2i\theta_\pm}$$
Si osservi che poichè $A,B,C,D$ dipendono dagli autostati $\psi_k$, anche le fasi $\theta_\pm$ dipenderanno dall'autovalore $k$.\\
Si vuole quindi cercare una stima numerica di $\theta\pm$ per determinare $\tau$ da:
    $$ (\tau \pm \rho) = e^{2i\theta_\pm} \Rightarrow \tau = 1/2(e^{2i\theta_+}+e^{-2i\theta_-})$$
    $$ \Rightarrow \tau^2 = sin^2(\theta_+ - \theta_-)$$
(due conti per dimostrarlo plis)
\\
Il coefficiente di trasmissione sarà qundi dato da:
    $$T = \tau^2$$

\subsection*{Discretizzazione numerica}
questioni sulla grid, cazzi e mazzi.
