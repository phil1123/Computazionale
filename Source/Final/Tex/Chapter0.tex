\textit{Si consideri la Hamiltoniana
$$ \Hh = -\frac{1}{2} \frac{ d^2}{dx^2} - V(x) $$
con potenziale assegnato da
$$
V(x)  = \begin{cases}
        0       & \mbox{se } x<-b \\
        4/b^2   & \mbox{se } -b \leq x \leq b \\
        0       & \mbox{se } x>b \\
         \end{cases}
$$
Sia $V0 = 4/b^2$. Siano:
$$
\begin{cases}
    \mbox{ZonaI} & = \{x<-b \} \\
    \mbox{ZonaII} & = \{-b \leq x \leq b\} \\
    \mbox{ZonaIII} & = \{x>b\}
\end{cases} $$
}%
\begin{enumerate}
  \item \textit{Diagonalizzare un'opportuna versione discreta di H con eig. Quindi cercare di
  estrarre il coefficiente di trasmissione dagli autovalore/vettori in funzione dell'energia
  o del numero d'onda, confrontando con il risultato analitico esatto.}

  \item \textit{Ripetere la procedura con un potenziale smooth come la barriera gaussiana e confrontare con il coefficiente di trasmissione calcolato come in classe (con il metodo
  dei pacchetti d'onda).}
\end{enumerate}

\textit{Come mai questo metodo non funziona con $V(x)$?}
